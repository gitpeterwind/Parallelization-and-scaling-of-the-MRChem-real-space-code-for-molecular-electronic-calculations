\documentclass[11pt]{article}
\usepackage{enumitem}
\usepackage{graphicx}
\usepackage{bm}
\usepackage{framed}
\usepackage{fullpage}
\usepackage[version=3]{mhchem}
\usepackage[usenames]{xcolor}
\definecolor{darkblue}{rgb}{0.0, 0.0, 0.55}
% for adjustwidth environment
\usepackage[strict]{changepage}
\usepackage{hyperref}

% for formal definitions
\usepackage{framed}

% environment derived from framed.sty: see leftbar environment definition
\definecolor{formalshade}{rgb}{0.95,0.95,1}

\newcommand{\ajwt}[1]{{\color{red}{#1}}}
\newenvironment{formal}{%
  \def\FrameCommand{%
    \hspace{1pt}%
    {\color{darkblue}\vrule width 2pt}%
    {\color{formalshade}\vrule width 4pt}%
    \colorbox{formalshade}%
  }%
  \MakeFramed{\advance\hsize-\width\FrameRestore}%
  \noindent\hspace{-4.55pt}% disable indenting first paragraph
  \begin{adjustwidth}{}{7pt}%
  \vspace{2pt}\vspace{2pt}%
}
{%
  \vspace{2pt}\end{adjustwidth}\endMakeFramed%
}
%\addtolength{\topmargin}{-.6in}
%\addtolength{\textheight}{.2in}
\begin{document}

Dear Editor,\\

We would like to thank the reviewers for their comments on our manuscript. 

Yours faithfully,
\\

Peter Wind

Magnar Bj{\o}rgve 

Anders Brakestad

Gabriel A. Gerez S.

Stig Rune Jensen 

Roberto Di Remigio Eik{\aa}s

Luca Frediani

\clearpage

{\Large \bf Response to Reviewers}

\vspace{1em}

{\bf Editorial Comments}

\begin{formal}
Please include an accessed by date for any reference citing a URL.
\end{formal}

{\bf Reviewer 1}

\begin{formal}
The article "The MRChem multiresolution analysis code for molecular electronic structure calculations: performance and scaling properties" by Wind and coauthors describes implementation and parallel performance of the MRChem program package. The paper is important and should definitely be published, but the presentation is not very concise and can be improved.

To me the article has too many implementation details, that distract from the core message of the article, which would be performance and scaling with system size.
\end{formal}

\begin{formal}
General remarks
 - Section 3 can be omitted entirely, in my opinion it is sufficient to state that the space is adaptively partitioned into 3d nodes/intervals. If necessary this chapter can be shortened and incorporated into Section 4
 \end{formal}
 
 \begin{formal}
 - I also feel that Sec 2.1 could be better included in Sec. 4
 \end{formal}
 
\begin{formal}
 - The level of detail is unnecessarily high in Section 4, e.g. explaining the "compressed" and "reconstructed" tree states
 \end{formal}
 
 \begin{formal}
 - The authors should always make clear what a "node" is: an MRA node (3d interval) or a compute node (possible also an MPI rank?). Also, they write about storing data "together" or "separately" which I assume refers to storing them on the same or different compute nodes? The same is true for statements that data are "locally accessible". This should be clarified.
 \end{formal}
 
 \begin{formal}
 - Sec 4.2: the naming "MW nodes in real space" confuses me, but I cannot come up with a better suggestion.
 \end{formal}
 
\begin{formal}
Minor remarks:
 - Sec 2.1: What is the difference between "Product of two MW functions" and "Application of a multiplicative operator"?
 \end{formal}
 
 \begin{formal}
 - Sec 2.1.1: derivatives are defined in a weak sense, but they are numerically not well-behaved.
 \end{formal}
 
 \begin{formal}
 - Sec 2.1.2 and 2.1.3: summing and applying does not commute: is there any numerical difference between both approaches?
 \end{formal}
 
 \begin{formal}
 - Eq. 17: $\sum_p$ instead of $\sum_k$
 \end{formal}

\begin{formal}
Suggestions:
 - The authors state correctly that data sparsity comes with MRA, even if orbitals have non-vanishing contributions from all regions. This is an important statement and has implications for both performance and precision and could be placed more prominently.
\end{formal}

\begin{formal}
 - The authors might mention that many first-order molecular properties, including gradients, can be computed at little extra cost via the Hellmann-Feynman theorem.
\end{formal}

\begin{formal}
 - It would be interesting to see scaling properties with respect to MPI ranks and compute nodes for a given molecule/precision.
\end{formal}



\clearpage

{\bf Reviewer 2}

\begin{formal}
The present manuscript is, according to the authors themselves, a ‘progress report’ to establish that the research ‘is going in the right direction’. Of course, most research publications can be considered as ‘progress reports’ and in the present case, the authors show that their MR implementation of the HF method is now competitive, or even superior, to existing GTO based methods for sufficiently large molecular systems. The manuscript discusses a number of technical points that are essential for achieving the computational performance, and this may be of interest to other groups developing code.
\end{formal}

We thank the reviewer for their nice summary of our work.

\begin{formal}
I find it slightly worrying that two well-established programs, LSDalton and ORCA, using a well-defined method (HF) and analytical integrals, give energy differences in the milli-H range (Table 5) for both the total and atomization energies. As I read the table, this is not due to the use of RI or ADMM, which is the ‘d’-labelled entry. Is it simply a result of a too loose integral screening factor? Perhaps this could be worth a comment.
\end{formal}

We did not investigate the origin of this discrepancy, as our point was to explore the performance of GTO-based and MW-based implementations using their default options. Tightening the integral screening factor and/or the SCF convergence threshold might resolve the discrepancy, but would most likely increase the computational cost.

\end{document}
